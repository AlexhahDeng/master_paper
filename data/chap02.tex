\chapter{相关研究综述}
当前,已经有大量关于云环境下隐私保护聚类方案的相关研究,并取得了长足进展。结合本文的主要研究内容,本章我们总结归纳了国内外关于隐私保护K-means聚类方案和隐私保护DBSCAN聚类方案的研究成果,深入分析方案各自在安全性、高效性以及可应用性方面存在的优势和问题,为本章后续研究内容进行铺垫。

\section{隐私保护K-means聚类方案}
近年来,在相关研究中使用的密码学工具主要可以分为两类,分别是同态加密和多方计算。本节将围绕基于同态加密的隐私保护K-means聚类方案以及基于多方计算的隐私保护K-means方案展开讨论。
\subsection{基于同态加密的K-means聚类}
Liu\cite{liu2014privacy}等人在全同态加密的基础上提出了一种外包聚类方案,由于密文不会保留距离的顺序特征,因此方案比较加密距离时需要用户提供陷门信息(trapdoor infomation),给用户带来巨大的开销。为了减少用户在聚类过程中的开销,Almutairi\cite{almutairi2017k}等人提出一种更新距离矩阵(UDM)的办法来提升比较密文距离的效率。但是上述两种方案都向云服务器泄漏了部分信息,例如簇大小、数据之间的距离以及簇中心的值。此外,上述两种方案中使用的全同态加密方法在论文\cite{wang2015notes}中证实不完全安全。外包计算中多用户上传数据存在单密钥限制以及低效频繁的客户端-服务器交流的问题,不利于大规模实际应用。为了解决上述问题,sok-102在Spark框架下基于双解密的公钥系统(PKCDD)提出了系列安全协议和外包聚类方案,但是该方案泄漏了中间结果,存在安全隐患。基于部分同态加密的协议通常涉及频繁的交互带来巨大的通信和计算开销,效率较低。对此,Wu等人\cite{wu2020secure}基于结合密文打包技术的全同态加密方案提出了安全高效的外包k-means聚类方案(SEOKC),通过对打包的密文进行并行计算,极大提升了方案的效率。然而协议中采用相同的噪声混淆数据后解密,泄漏了数据的分布信息存在安全性问题。

上述方案均没有满足语义安全性,为了解决这一问题。Rao等人\cite{rao2015privacy}在Paillier加密方案的基础上提出了一个外包k-means聚类方案。他们考虑到多用户场景下,用户一旦上传数据后无需进行任何操作。在保护数据安全的同时,方案向两个不共谋的云服务器隐藏了数据访问模式。然而,在密文上执行大量交互协议引入了巨大的计算开销,使得方案无法在大型数据集上应用。Kim等\cite{kim2018privacy}人认为方案\cite{rao2015privacy}之所以低效是因为基于比特数组的比较和随机选择初始簇中心,为了解决这个问题,他们提供了一种全新的能够在密文上进行快速比较的方案并且根据数据分布来选簇中心,该方案耗时相较于论文\cite{rao2015privacy}提升3倍。为了解决全同态中计算除法的问题,论文\cite{jaschke2019unsupervised}提出了一种自然编码方式,但是该方式效率非常低,因此作者对K-means方案进行修改避开除法以适应全同态方案的特点,最后为了进一步提升效率,作者以精确性为代价提出了一种近似的隐私保护K-means方案。
\subsection{基于多方计算的K-means聚类}
基于多方计算的隐私保护K-means聚类方案通常使用秘密共享和混淆电路两种基础工具来构造协议。Doganay等人\cite{doganay2008distributed}提出了基于加性秘密共享的分布式隐私保护K-means聚类方案,相较于之前的工作效率有所提升,但是方案最后向所有参与方揭露了簇划分的信息。大部分相关研究通常基于半诚实模型,对此,Patel等人\cite{patel2012efficient}\cite{patel2013privacy}等人分别在半诚实模型下基于Shamir秘密共享设计,在恶意模型下基于零知识证明提出了两种不同的聚类方案。Upmanyu等人\cite{upmanyu2010efficient}和Baby等人\cite{baby2016distributed}基于中国剩余定理(CRT-SS)分别提出了分布式阈值秘密共享方案。然而,上述方案均泄漏了聚类相关信息,因此存在数据安全问题\cite{hegde2021sok}。在2020年,Mohassel等人\cite{mohassel2019practical}等人提出了一种两方k-means聚类方案,基于2-out-of-2加性秘密共享。虽然该方案保证了数据安全,但是协议的构造主要依赖混淆电路和不经意传输,带来大量的计算和通信开销,因此效率低下,不适合海量数据上的大规模的聚类任务。

此外,有许多相关研究将同态加密和多方计算结合起来设计协议,以结合二者的优势。Vaidya等人\cite{vaidya2003privacy}提出首个基于安全多方计算的隐私保护k-means聚类方案\cite{meskine2012privacy},应用于垂直划分的数据集,借助了同态加密、安全混淆\cite{du2001privacy}以及混淆电路密码学技术,但是对每个数据进行都需要进行安全混淆使得效率较低。Jagannathan等人\cite{jagannathan2005privacy}提出将除法转换为乘以倒数,但是这种方式不满足准确性,无法正确实现k-means算法。举个简单的例子,在$\mathbb{Z}_{21}$上用11除5,结果应该是约等于2,但是$11*5^{-1}=11*17=19$,同时该方案计算开销巨大\cite{bunn2007secure}。Su等人\cite{su2007privacy}基于不经意多项式评估以及安全近似工具构建了相关方案,实现了安全数据标准化,但是除法操作存在问题,无法提供完整安全性。Sakuma等人\cite{sakuma2010large}基于paillier加密系统,姚式混淆电路设计了一个能够用于混合数据划分数据集的隐私保护聚类方案,该方案可扩展能容错,但是泄漏了结果中簇包含数据个数的信息。Jiang等人\cite{jiang2020efficient}提出了一个两方外包隐私保护k-means聚类方案,该方案需要云平台与用户进行多轮交互,同时更新的簇中心会被泄漏,存在安全性问题。

Bunn等人\cite{bunn2007secure}基于同态加密和加性秘密共享设计了一系列协议,实现了完全安全的隐私保护壳k-means聚类。解决了安全多方计算中除法的计算问题,提出了一个新的协议来随机选择k个初始簇中心。但是开销巨大。Mohassel等人\cite{mohassel2019practical}基于混淆电路、秘密共享和1-out-of-n 不经意传输提出了一个兼顾安全与高效的两方隐私保护聚类方案,作者结合k-means算法的特点,需要计算一个固定点到多个簇中心的距离,设计了快速计算的协议提升效率。

\section{隐私保护DBSCAN聚类方案}
% 我的建议是直接搬运2021年bozdemir的文章相关工作了!
过去关于隐私保护聚类的研究大多是基于k-means展开,鲜少有关于隐私保护DBSCAN的相关研究。这里我们综合过去数十年的研究成果进行总结分析。

Anikin和Gazimov\cite{anikin2017privacy}在半诚实模型下,针对垂直划分数据集,设计了一种允许任意多个参与方加入的隐私保护DBSCAN协议。协议主要依赖于同态加密和加性秘密共享,加密操作的开销巨大。此外,聚类要求所有参与方都保持在线状态,因此难以应用到外包计算场景中。一个参与方负责主要的计算工作,并获取所有数据之间的明文距离信息,簇划分情况以及簇大小。作者既没有给出有说服力的协议实现也没有给出性能分析。

\cite{2006Privacy,jiang2008privacy,kumar2007privacy,liu2012privacy,xu2007protocols}针对垂直划分或者水平划分的数据集,而\cite{almutairi2018secure,liu2012privacy}则能够处理任意划分类型的数据集。上述方案用到了同态加密、随机混淆。部分方案引入了第三方来保护数据隐私安全。但是,所有方案均泄漏了聚类信息,例如簇大小,数据相邻关系\cite{almutairi2018secure,jiang2008privacy,kumar2007privacy,liu2012privacy,rahman2017towards}或数据之间的距离\cite{2006Privacy}。在特定的情况下,泄漏这样的信息会使得原始数据被\cite{kumar2007privacy,liu2012privacy}。在论文\cite{xu2007protocols}中,当数据被划分到所属簇时,原始信息会被揭露。此外,方案均没有给出实现和实验性能分析。

Rahman等人\cite{rahman2017towards}借助同态加密技术设计了一个外包场景下的隐私保护DBSCAN协议,不受信的服务器在数据持有者的帮助下执行聚类过程。然而, 该方案泄漏了簇大小和数据相邻关系信息给服务器。并且,作者没有对方案进行详细的实验分析。Bozdemir等人\cite{2021Privacy}在半诚实模型下基于ABY\cite{2015ABY}框架提出了一个隐私保护DBSCAN聚类方案,该方案既可以作为两方计算协议也可以用于外包计算场景,同时协议设计引入了并行计算提升效率。然而方案中迭代深度需要人工设定,对开销影响较大。同时,向服务器揭露1比特数据来判断是否满足迭代终止条件。Wang等人\cite{wang2022homomorphic}针对数据准确性和计算开销的不同要求,提出了三种数据预处理方案,基于同态加密设计了一种用户与云平台之间进行安全比较的协议。文章的实验部分显示提升了聚类效率,但是数据集较少,分析不够全面。

\cite{2018DP,2015A}中,作者借助差分隐私技术来保护用户的隐私数据安全。在这些研究中,数据持有者进行聚类,不受信的服务器通过请求访问聚类结果。然而,该协议无法应用于数据来源于多个用户的场景。此外,每个参数维度添加到距离的噪声大小取决于具体的隐私要求,而且会不可避免的影响数据的实际意义,最终影响到聚类结果的质量。

综上所述,隐私保护DBSCAN方案中存在的问题主要可以分为如下几类:数据持有者高度参与聚类过程,不适合外包计算场景;基于差分隐私,牺牲准确性;不完全安全,泄漏部分信息;引入大量密码学技术,协议效率较低没有扩展性。

\section{本章小节}
本节主要分为两部分,第一部分归纳总结了隐私保护K-means聚类方案的国内外相关研究,基于同态加密和多方计算两种技术分别展开讨论。第二部分归纳总结了隐私保护DBSCAN的国内外相关研究,概述了最近数十年的研究发展,并探讨了研究中存在的部分问题。在表格\ref{c2-table-compare}中,我们给出了所有完全安全的隐私保护方案以及简要的介绍信息。
总体而言,当前云环境下隐私保护聚类方案的设计,需要权衡安全性、效率和数据可用性。本文将围绕k-means和DBSCAN展开云环境下的隐私保护聚类研究。

\begin{table}[htbp]
	\centering	
	\renewcommand{\arraystretch}{0.9}
	\caption{完全安全的隐私保护K-means和DBSCAN方案}
	\label{c2-table-compare}
	\scalebox{1.0}{
		\begin{tabular}{c|c|c|c|c}% 通过添加 | 来表示是否需要绘制竖线c|
			\hline  % 在表格最上方绘制横线
			算法&论文&隐私计算工具&场景&数据集划分\\%&SKD
			\hline % 在表格最下方绘制横线
			K-means&\cite{bunn2007secure}&HE+ASS&两方&任意方式\\
			&\cite{rao2015privacy}&HE&双服务器外包&水平划分\\
			&\cite{jaschke2019unsupervised}&HE&单服务器外包&-\\
			&\cite{kim2018privacy}&HE&双服务器外包&-\\
			&\cite{mohassel2019practical}&GC&双服务器外包或两方&水平划分\\
			\hline
			DBSCAN&\cite{zahur2013circuit}&GC&两方&水平划分\\
			&\cite{bozdemir2021privacy}&GC+ASS&双服务器外包或两方&任意划分\\
			\hline
		\end{tabular}	
	}
\end{table}