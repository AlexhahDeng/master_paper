\chapter{总结与展望}
\section{工作总结}
云环境下的聚类应用赋能大数据挖掘与分析,为人类生活与工作带来了诸多便利,解放了资源有限用户的生产力,让用户可以关注分析与发现。然而,伴随而来的数据安全问题也引发了广泛的关注和重视。拥有数据的用户在享受便捷服务的同时,上传的隐私数据与聚类获取的划分结果可能会被泄漏。过去几年来,因为数据泄漏而引发的问题不胜枚举,用户对于提供云服务的公司信心受损,公司蒙受巨大损失。为此,本文针对隐私保护聚类方案中存在的安全问题、效率问题以及准确性问题展开研究。本文的主要研究内容和贡献包括以下几个方面:

\begin{compactitem}
	\item 
	提出了一种基于Kd-tree的隐私保护外包K-means聚类方案。针对目前隐私保护K-means聚类中存在的无法兼顾安全和高效的问题,设计了一种全新的解决方案。方案基于秘密共享技术,在不共谋的双云服务器协作的场景下,设计了一系列可扩展可迁移的安全计算协议。通过构造Kd-tree数据结构加速聚类过程,减少了冗余计算,加速算法收敛。方案能够迁移到大型数据集上,在即将收敛时,效率远远超过传统隐私保护K-means方案。在详细的安全分析的基础上,通过实验验证,我们认为方案不仅能够在兼顾效率和安全的情况下完成聚类,同时还保证了结果的准确性。
	\item 
	提出了隐私保护DBSCAN系列方案。针对隐私保护DBSCAN相关研究中存在的效率较低、划分结果不稳定以及依赖人工参数设置等问题分别提出了对应的解决方案。在双云服务器交互的场景下,多用户秘密共享数据后分别上传给云平台,通过运行系列安全计算协议,获取最终的聚类结果并还原。所述方案不仅可以应用于外包场景还能够迁移到多方计算的场景中,具备可扩展性。经过全面的实验分析,证明方案一兼顾效率与安全性,方案二在保证高效与安全的同时提升了聚类质量,方案三能够在特定数据集上发挥较好的效果。
\end{compactitem}

\section{研究展望}
目前云环境下隐私保护聚类方案已有较多相关研究,但是大多聚类方案只适用于数据量较小的情况,难以应用于需要进行实时分析的大数据场景。由于自身能力有限与时间不足,对相关问题的研究不够深入。因此在本文研究的基础上,接下来可以在如下两个方面进一步研究探索:

% 近似隐私保护聚类方案
本文设计的隐私保护聚类方案虽然能够在兼顾效率与安全的情况下,在中小规模数据集上获取高质量的聚类结果,但是在海量数据上聚类耗时较长,无法满足实际应用的需求,因此存在较大局限性。由于数据挖掘分析本身的特点,允许在聚类精度损失可接受的情况下对算法进行修改,获取近似结果以减少计算量提升效率、扩展性和易用性。同时,随着云计算和大数据的进一步发展,如何高效的对海量数据进行隐私保护挖掘分析具有较大的现实意义。

% 其他聚类算法的隐私保护方案
目前关于云环境下的隐私保护聚类研究大多基于K-means算法,其次是DBSCAN算法和层次聚类算法。聚类算法还包含很多其他分支,例如BIRCH算法聚类速度快能够识别噪声点、GMM算法获取划分到每个类的概率以及Affinity Propagation算法对于初始值不敏感等等。上述聚类算法分支都有其适用的场景和优缺点,但是相关隐私保护方案则研究较少。对于这些算法设计相关隐私保护方案极具现实意义,能够进一步扩大云环境下聚类算法的应用以及提升聚类的效果。