\begin{resume}
\ifreview

\ifismaster
该论文作者在学期间取得的阶段性成果(学术论文等)已满足我校硕士学位评阅相关要求。为避免阶段性成果信息对专家评价学位论文本身造成干扰,特将论文作者的阶段性成果信息隐去。
\else
该论文作者在学期间取得的阶段性成果(学术论文等)已满足我校博士学位评阅相关要求。为避免阶段性成果信息对专家评价学位论文本身造成干扰,特将论文作者的阶段性成果信息隐去。
\fi

\else

\ifisresumebib
%该论文作者在学期间取得的阶段性成果(学术论文等)已满足我校硕士学位评阅相关要求。为避免阶段性成果信息对专家评价学位论文本身造成干扰,特将论文作者的阶段性成果信息隐去。
	\begin{refsection}[ref/resume.bib]
	\settoggle{bbx:gbtype}{false}%局部设置不输出文献类型和载体标识符
	\settoggle{bbx:gbannote}{true}%局部设置输出注释信息
	\setcounter{gbnamefmtcase}{1}%局部设置作者的格式为familyahead格式
	\nocite{ref-1-1-Yang,ref-2-1-杨轶,ref-3-1-杨轶,ref-4-1-Yang,ref-5-1-Wu,ref-6-1-贾泽,ref-7-1-伍晓明}
	
	\setlength{\biblabelsep}{12pt}
	\printbibliography[env=resumebib,heading=subbibliography,title={发表的学术论文}] % 发表的和录用的合在一起

	\end{refsection}


	\begin{refsection}[ref/resume.bib]
	\settoggle{bbx:gbtype}{false}%局部设置不输出文献类型和载体标识符
	\settoggle{bbx:gbannote}{true}%局部设置输出注释信息
	\setcounter{gbnamefmtcase}{1}%局部设置作者的格式为familyahead格式
	\nocite{ref-8-1-任天令,ref-9-1-Ren}%
	
	\setlength{\biblabelsep}{12pt}
	\printbibliography[env=resumebib,heading=subbibliography,title={研究成果}]

	\end{refsection}

\else

%  \section*{发表的学术论文} % 发表的和录用的合在一起
%
%  \begin{enumerate}[label={[\arabic*]}]
%
%  \end{enumerate}
%
%  \section*{研究成果} % 有就写,没有就删除
%  \begin{enumerate}[label=\textbf{[\arabic*]}]
%  \addtolength{\itemsep}{-.36\baselineskip}%
%  \item 任天令, 杨轶, 朱一平, 等. 硅基铁电微声学传感器畴极化区域控制和电极连接的
%    方法: 中国, CN1602118A. (中国专利公开号.)
%  \item Ren T L, Yang Y, Zhu Y P, et al. Piezoelectric micro acoustic sensor
%    based on ferroelectric materials: USA, No.11/215, 102. (美国发明专利申请号.)
%  \end{enumerate}
\fi
\fi
\end{resume}
