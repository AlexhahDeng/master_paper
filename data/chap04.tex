\chapter{隐私保护密度聚类方法研究}
\section{引言}
% 1.说明隐私保护k-means研究存在的问题
海量可获取的数据以及云计算提供的计算能力驱动机器学习的快速发展。有监督学习例如神经网络,使用带标签的数据来训练具有识别能力的模型。与之相反的是无监督学习,没有训练模型的过程,旨在从未标记的数据中发现未知的模式和特征。聚类是一种广泛使用的无监督机器学习工具,能够将相似的数据划分到同一个集合中。实际应用中,聚类常用于许多数据安全极为重要的领域,例如商业数据分析,市场数据挖掘以及医院病理信息分析等。上述场景中,数据一旦泄露会造成极大经济损失。

目前,已有许多关于隐私保护聚类方案的研究,其中数量最多的是与k-means相关的研究ref-sok。然而k-means聚类过程相对比较简单,并且只能检测到凸型簇,因此适用数据集类型极为受限。此外,簇的数量$k$必须要根据专业领域知识提前给出,如果只了解数据集的子集难以确定$k$的值。同时,k-means聚类不包含噪声的概念,并且聚类的结果对异常值非常敏感因为每一个输入都要被划分到某一个簇中。因此,即便某个输入不属于任何一个簇,它都会被划分到距离最近的簇,影响该簇的中心位置(簇中所有数据的平均值)。

% 2.隐私保护dbscan的好处
为了解决上述隐私保护方案中存在的问题,这里我们引入对隐私保护density-based spatial clustering of applications with noise(dbscan)的研究。dbscan是一种由ref-dbscan-ester20提出的更加灵活的聚类算法,能够检测到任意形状的簇。此外,簇的数量是根据数据集的特点灵活产生的,无需人工确定。该算法对异常值不敏感,会将其标记为噪声。本章首先提出了方案一,一种安全高效的隐私保护dbscan方案,该方案针对明文算法进行改进以适应安全计算的特点,将复杂度降低为$O(n^2)$,显著降低聚类所需时间。

传统dbscan方案存在数据划分结果不稳定的问题,即最终划分结果取决于算法运行中数据遍历的顺序,将数据打乱后重新进行聚类,同一个点可能会被划分到不同的簇中。为了解决该问题,在ref-redbscan论文思想的基础上,我们在方案二中提出了改进的隐私保护dbscan,以获取稳定的聚类结果。

尽管dbscan具有诸多优点,其聚类过程涉及两个重要参数MinPts和Eps,这两个参数的取值与数据分布密切相关,通常需要人工分析后给出。为解决该问题,在论文ref-dbhc的基础上,我们在方案三中提出了基于dbscan的隐私保护层次聚类,借助k近邻算法和k线图来决定聚类参数,并进行多轮聚类来适应不同密度的簇。
% 3.本章组织结构
本章的组织结构如下:第\ref{s4-yubei}节介绍了dbscan的相关内容。第\ref{s4-wenti}节中描述了系统模型、安全模型以及设计目标。第\ref{s4-subpro}节中增加了一些基于秘密共享的隐私保护计算模块。第\ref{s4-t1}节对隐私保护dbscan方案进行了详细介绍。第\ref{s4-t2}节中提出了方案二改进的隐私保护dbscan方案。第\ref{s4-t3}节中阐述了基于dbscan的隐私保护层次聚类方案。第\ref{s4-lilun}节中从理论上分析了方案的正确性和安全性。第\ref{s4-shiyan}节中对方案进行了实验评估。最后,在\ref{s4-xiaojie}节对本章进行了总结。
\textbf{}
\section{预备知识}
本节我们主要介绍DBSCAN的明文算法以及相关的改进算法。
\label{s4-yubei}
\subsection{DBSCAN}
Density-based Spatial Clustering of Applications with Noise(DBSCAN)是一种基于密度的聚类算法,在密集区域聚集在一起的数据点被划分到同一个簇中,稀疏区域的数据被标记为噪声\cite{khan2014dbscan}。算法要求两个重要参数:
\begin{compactitem}
	\item $\epsilon$: 确定两个点被视为相邻的最大距离
	\item MinPts: 确定领域内至少包含多少点才能构成一个簇
\end{compactitem}

DBSCAN围绕每个数据点以$\epsilon$为半径构建圈,并将其划分为核心点、边界点以及噪声点。若圈内不包含任何点,则认为是噪声点。若数据点圈内包含点的数量至少为MinPts个,则认为是核心点,否则为边界点。围绕上述定义展开,DBSCAN还包含如下概念:

假设样本集为$D=(x_1,x_2,...,x_m)$:
\begin{compactitem}
	\item 密度直达: 若$ x_i $位于$ x_j $的$ \epsilon- $邻域内,且$ x_j $为核心对象,则称$ x_i $由$ x_j $密度直达,反之不一定成立。
	\item 密度可达: 对于$ x_i $和$ x_j $,若存在样本序列$ p_1,p2,...,p_t $,满足$ p_1=x_i,p_T=x_j $,且$ p_{t+1} $由$ p_t $密度直达,则称$ x_j $由$ x_i $密度可达。
	\item 密度相连: 对于$ x_i $和$ x_j $,如果存在核心点$ x_k $,使得$ x_i $和$ x_j $均由$ x_k $密度可达,则称$ x_i $和$ x_j $密度相连。
\end{compactitem}


下面我们具体阐述DBSCAN聚类算法的流程:

\begin{enumerate}
	\item 初始化核心点集合$ \Omega = \varnothing  $,初始化簇数量$ k=0 $,初始化未访问点集合$ \Gamma = D $,簇划分$ C=\varnothing $
	\item 遍历所有点$ i=1,2,...,m $,按如下方式找到所有核心点:
	      \begin{enumerate}
		      \item 计算距离,找到样本$ x_i $的$ \epsilon- $邻域内所有点集合$ N_{\epsilon}(x_i) $
		      \item 如果集合包含数据点个数满足$ |N_{\epsilon}(x_i)| \geq MinPts $,将$ x_i $加入核心点集合:$ \Omega = \Omega \cup \{x_j\} $
	      \end{enumerate}
	\item 若$ \Omega = \varnothing $,则算法结束,否则转入步骤4
	\item 在$ \Omega $中,随机选择一核心点$ o $,初始化当前簇包含核心点集合$ \Omega_{c}=\{o\} $,簇序号为$ k=k+1 $,当前簇包含点集合$ C_k=\{o\} $,更新未访问样本集合$ \Gamma = \Gamma - \{o\} $
	\item 若$ \Omega_{cur} = \varnothing $, 则簇$ C_k $聚类完毕,更新簇集合$ C=\{C_1,C_2,...,C_k\} $,更新核心点集合$ \Omega = \Omega - C_k $
	\item 从$ \Omega_{cur} $中取出一个核心点$ o' $,通过邻域距离$ \epsilon $找到所有$ \epsilon- $邻域子集$ N_{\epsilon}(o')  $,令$ \Delta = N_{\epsilon}(o') \cap \Gamma $,更新当前簇包含点集合$ C_k = C_k \cup \Delta $,更新未访问点集合$ \Gamma = \Gamma - \Delta $,更新$ \Omega_{cur} = \Omega_{cur}\cup(\Delta\cap\Omega)-o' $,转入步骤5
	\item 输出结果为: 簇划分$ C=\{C_1,C_2,...,C_k\} $
\end{enumerate}
%来源为https://www.cnblogs.com/pinard/p/6208966.html

DBSCAN能够应用于任意维度的数据集。铭文上最坏情况下的复杂度为$ O(n^2) $,其中$ n $为数据的数量。

\subsection{改进的DBSCAN}
正如提出DBSCAN的论文\cite{ester1996density}中所说,该算法在检测相邻簇的边界数据点时划分结果不稳定。若相邻簇$ C_1 $与$ C_2 $的边界上有一数据$ x $,按照划分规则,该点既可以被划分到$ C_1 $,也可以被划分到$ C_2 $,最终划分结果取决于算法运行时$ x $被处理的顺序。

在\cite{tran2013revised}中,作者认为边界对象对于传统DBSCAN算法簇的扩展没有贡献,因此可以改造算法的形式,以使得边界点的划分结果稳定。文章提出在聚类过程中,暂时不处理边界对象,直到所有核心对象都被分配到对应的簇中。最后,将所有未被划分的边界对象划分到最近的核心对象所属簇中即可。
%todo 可以来个图展示一下
\subsection{基于DBSCAN的层次聚类}
传统DBSCAN算法中,MinPts和$ \epsilon $的取值对于聚类效果有极大影响,但同时又难以确定,因为二者受数据分布的影响较大。为了解决这个问题,论文\cite{latifi2021dbhc}提出一种基于DBSCAN的层次聚类方案。首先借助$ k $近邻算法和$ k $线图来决定两个参数。因为实际数据大多不是均匀分布的,需要计算不同的$ \epsilon $值。然后,根据不同的$ \epsilon $值来多次聚类。最后,若最终结果所得簇个数超过了用户实际认可的数量,则选择不同的簇进行合并,直到满足要求。该方案经过实验,取得了较好的效果。
\section{问题描述}
\label{s4-wenti}
\subsection{系统模型}
\begin{figure}[htbp]
	\centering
	\includegraphics[width=3in]{img/ch4-sysmod.png}%width=\linewidth
	\caption{System Model}
	\label{s4-sysmod}
\end{figure}
如图\ref{s4-sysmod}所示,我们的系统由两个部分组成:服务器端和用户端。与图\ref{sys model}类似,服务器端为两个云服务器,获取用户数据后执行一系列安全协议,最后分发结果。

不同的是,图\ref{sys model}中客户端为某一个持有全部明文数据的组织机构,本方案系统模型中的用户既可以是单一机构,也可以是各自持有不共享数据的不同机构。第二种场景下,用户可以是不同的医疗机构,期望能够在不泄露患者信息的情况,联合起来对医疗诊断或者症状信息进行聚类来做数据挖掘与分析。用户将数据进行秘密共享后发送给双云服务器即可。在获取到聚类结果后,不同机构各自执行简单的算法来还原结果。
\subsection{安全模型}
本方案基于半诚实模型的假设,细节说明见\ref{s3-anquanmoxing}中的说明。具体来说:

对用户而言,不同用户可能会根据聚类结果,推断其他用户输入数据的具体内容。

对服务端而言,可能会试图根据秘密共享的数据、中间计算结果以及聚类结果推测原始内容。
\subsection{设计目标}
我们的目标是设计三种针对不同场景的隐私保护DBSCAN方案,具体而言,设计目标如下:

\begin{compactitem}
	\item 能够高效的在密文上运行朴素的DBSCAN算法,在效率略微降低的前提下,运行结果稳定的隐私保护DBSCAN算法。在运行效率可接受的情况下, 运行参数无需提前设置的基于DBSCAN的层次聚类。
	\item 三个密文方案运行结果与对应明文方案一致,精度无损失。
	\item 云服务器上所有数据均在密文状态,并且无法被利用来还原初始数据信息。不同用户之间无法知悉彼此的数据和聚类结果。
\end{compactitem}
\subsection{系统输入}
数据持有者首先利用加性秘密共享来将数据划分为两份,分别发送给两个云服务器。采用加性秘密共享分发数据的方式允许系统中加入任意多个数据持有者提供聚类数据。此外,我们的系统能够支持任何数据划分方式,例如水平划分、垂直划分以及混合划分。水平划分方式指的是,不同用户分别持有完整但是互异的数据\cite{gheid2016efficient},垂直划分方式则是指不同用户持有相同数据记录的不同参数\cite{doganay2008distributed},混合划分方式则是上述两种方式的结合\cite{yu2010multi}。
\section{基于秘密共享的隐私保护计算模块}
\label{s4-subpro}
在\ref{s3-mokuai}节所述算法的基础上,我们根据隐私保护DBSCAN方案的特点设计了如下计算模块。
\subsection{安全排序协议}
安全排序协议主要由安全比较协议构成,用户输入秘密共享的数组后,我们期望能够得到按照升序排列的密文结果。协议的核心思想为,将$ n $个数据上的排序拆分为$ n $次求解序列最值。

具体如算法\ref{alg_sort}所示,我们进行$ n $次迭代,2行开始求解最小值,结果为秘密共享值$ \langle t \rangle_i \in \{0,1\}, i\in[1,n] $,数组中最小值的位置为1,其余均为0。4-6行开始遍历每个结果,$ res $通过累加记录最小值的具体值。6行的目的是更新原始数组,通过将最小值变为全局最大值,以消除对后续计算的影响。若当前元素为最小值(即$\langle t \rangle_i=1  $ ),则对应$ D[i] $中的值变为最大值maxv,若不是则保留原值,不做修改。
\begin{algorithm}[htbp]
	\renewcommand{\algorithmicrequire}{\textbf{输入:}}
	\renewcommand{\algorithmicensure}{\textbf{输出:}}
	\caption{安全排序协议}
	\label{alg_sort}
	\begin{algorithmic}[1]
		\REQUIRE $ D = \{\langle x\rangle_1, \langle x\rangle_2,...,\langle x\rangle_n\} $
		\ENSURE sorted result$ D^{\prime}\{\langle x^{\prime}\rangle_1, \langle x^{\prime}\rangle_2,...,\langle x^{\prime}\rangle_n\} $
		\FOR{$ i=1 $ to $ n $}
		\STATE $ \{\langle t\rangle_1,...\langle t \rangle_n\} \leftarrow smin(D)$
		\STATE $ res \leftarrow 0 $
		\FOR {$ j=1 $ to $ n $}
		\STATE $ res \leftarrow res + mul(\langle t \rangle_j, D[j]) $
		\STATE $ D[j] \leftarrow mul(D[j],1-\langle t \rangle_j) + mul(maxv, \langle t \rangle_j)$
		\ENDFOR
		\STATE $ r[i] \leftarrow res $
		\ENDFOR
	\end{algorithmic}
\end{algorithm}
\section{隐私保护dbscan方案}
\label{s4-t1}
本节提出的隐私保护DBSCAN方案主要分为三个阶段,具体介绍如下:
\begin{compactitem}
	\item 初始化:通过获取的密文数据,初始化存储在云服务器上的自定义数据结构体(以下简称元素),然后计算不同元素之间的距离关系,并判断元素是否为核心点。
	\item 聚类:第一阶段获取的信息,我们提出了一种全新的聚类方法,引入临时簇这一概念,通过记录临时簇是否相连的信息来聚类。
	\item 还原结果:用户获取聚类记录的临时簇相连信息,通过深度优先遍历算法来还原数据簇划分结果。
\end{compactitem}

\subsection{初始化}
% 首先介绍input element
对于每个秘密共享的数据,我们构造sharedElem对象。$ cluId, cluId \in [1, n]$为初始簇中心,值取该数据在数组中的下标,比如第一个数据初始化clusterId为1。$ isMark, isMark\in [0,1] $表示该数据对象是否已经被划分到簇中。data则存储对应的秘密共享值,为一个向量。最后$ isCore ,isCore\in[0,1]$标识数据对象是否为核心点。
\begin{algorithm}
	\begin{algorithmic}[1]
		\STATE \textbf{class}sharedElem(sdata,idx):
		\STATE \hspace{\algorithmicindent} cluId = idx
		\STATE \hspace{\algorithmicindent} isMark = 0
		\STATE \hspace{\algorithmicindent} data = sdata
		\STATE \hspace{\algorithmicindent} isCore = 0
	\end{algorithmic}
\end{algorithm}

构造完sharedElem对象后,我们得到了sharedList数组。在初始化阶段,我们计算对象之间的距离,获取相邻关系,同时判断是否为核心点。具体过程如算法\ref{alg_ini}所示。1行,我们构造数组$ cnt $存储每个元素邻域范围内包含其他元素的个数。2-7行,我们计算距离信息,将第$ i $个元素与其他所有元素的距离$ d_j,j\in[n] $与$ eps $相比较,判断二者是否相邻,结果存储到二维数组$ u_{ij},i,j\in[n] $中,最后统计元素$ i $邻域范围内包含点的个数。9行,我们并行比较邻域点个数$ cnt $与$ minpts $的大小关系,并更新sharedList的$ isCore $属性。
\begin{algorithm}[htbp]
	\renewcommand{\algorithmicrequire}{\textbf{输入:}}
	\renewcommand{\algorithmicensure}{\textbf{输出:}}
	\caption{初始化}
	\label{alg_ini}
	\begin{algorithmic}[1]
		\REQUIRE 数组sharedList
		\ENSURE 相邻关系二维数组$ u $,更新后的sharedList
		\STATE $ cnt \leftarrow [0,...,0]_n $
		\FOR {$ i=1 $ to $ n $}
		\FOR {$ j=1 $ to $ n $}
		\STATE $ d_{j} \leftarrow SED(sharedList[i], sharedList[j]) $
		\ENDFOR
		\STATE $ u_i \leftarrow sc(d, [eps,...,eps]_n) $
		\STATE $ cnt \leftarrow sum(u_{ij})$ for $j \in [n] $
		\ENDFOR
		\STATE $ r \leftarrow sc(cnt, [minpts]_n) $
		\FOR {$ i=1 $ to $ n $}
		\STATE sharedList[i].isCore = r[i]
		\ENDFOR
	\end{algorithmic}
\end{algorithm}

\subsection{聚类}
初始化之后,我们获得了元素的$ isCore $信息,以及元素之间的相邻关系$ u_{ij},i,j\in[n],u_{ij}\in\{0,1\} $。接下来,我们将具体介绍一种基于DBSCAN的全新聚类方式。该算法的核心在于构造若干临时簇,通过记录临时簇之间的连接关系,最后由用户来还原最终簇划分结果。

在初始化的过程中,我们默认每个元素的初始$ cluId $为下标,即每个元素都默认为独立的临时簇。在聚类的过程中,对于核心点,我们修改其$ \epsilon $范围内所有未标记点的$ cluId $,并将状态修改为已标记。对于所有非核心点,无法修改其他点的属性,并且被标记后,无法再被其他核心点标记。经过上述过程,数据集已经被划分为若干临时簇。根据DBSCAN的原理可知,单个簇由若干密度可达的核心点展开。通过记录核心点之间的相邻关系,我们在还原结果的阶段,将相连的临时簇合并,即可得到最终的结果。

具体流程如\ref{alg_clu1}所示,为了简化说明,我们将输入数组简写为$ d $,并将结果记录在plist数组中,数组中每个元素的结构依次为,元素$ i $的$ cluId $,元素$ j $的$ cluId $,临时簇1和簇2是否相连。2行,更新元素$i$的isMark标识,若元素$i$是核心点($isCore=1$)或者已经被划分($isMark=1$),则令$ d[i].isMark $的值为$ 1 $,这里的计算逻辑体现的是逻辑或关系。3-8行,我们开始对其他所有元素进行处理。4-6行的计算逻辑是,若元素$ i $与元素$ j $相邻($ u_ij = 1 $),且元素$ j $未被标记过($ d[j].ismark=0 $),同时元素$ i $为核心点($ d[i].isCore =1 $),则元素$ j $的簇标识修改为与元素$ i $的簇标识一致,否则不做修改。7行,记录临时簇相连信息,若$ d[i] $和$ d[j] $同为核心点,并且相邻$ u_{ij}=1 $则二者相连。8行,我们更新$ plist $的信息,记录对应值。

\begin{algorithm}[htbp]
	\renewcommand{\algorithmicrequire}{\textbf{输入:}}
	\renewcommand{\algorithmicensure}{\textbf{输出:}}
	\caption{聚类}
	\label{alg_clu1}
	\begin{algorithmic}[1]
		\REQUIRE 数组sharedList简写为d,相邻关系$ u_{ij},i,j\in[n] $
		\ENSURE 簇连接关系$ plist $
		\FOR{$ i=1 $ to $ n $}
		\STATE $d[i].isMark\leftarrow d[i].isCore+d[i].isMark$\\ $-mul(d[i].isCore,d[i].isMark)$
		\FOR {$ j=1 $ to $ n $}
		\STATE $ m_1 \leftarrow mul(d[i].isCore, u_{ij}) $
		\STATE $ m_2 \leftarrow mul(1-d[j].isMark, m_1) $
		\STATE $ d[j].cluId \leftarrow d[j].cluId + mul(m_2, d[i].cluId - d[j].cluId)$
		\STATE $ d[j].isMark \leftarrow d[j].isMark+m_1 - mul(d[j].isMark, m_1) $
		\STATE record new element in plist $ (d[i].cluId, d[j].cluId, mul(m_1, d[j].isCore)) $
		\ENDFOR
		\ENDFOR
	\end{algorithmic}
\end{algorithm}

\subsection{还原聚类结果}
\label{task1-huanyuan}
聚类后,我们在plist中记录了所有临时簇的相邻关系。云服务器分别将自己持有的份额分别发送给用户还原结果。在\ref{alg_huanyuan1}中,我们详细说明如何还原结果。1-5行,我们构造大小为$ n \times n $的全零二维数组,来存储邻接关系,若相邻,则对应值大于零。6行,我们初始化一个数组来标记每个数据是否已经被划分,若已经被划分,则不再继续处理。7行,我们构建最终结果的簇编号,自1开始累加。8行开始遍历,若元素$ i $未被划分,则更新mark,开始构建新簇,执行深度优先遍历找到所有与$ i $相连的元素纳入到新簇中。dfs的过程如\ref{alg_huanyuan2}所示,边递归边记录映射关系。最后,根据映射关系,我们更新所有元素的最终簇编号。

\begin{algorithm}[htbp]
	\renewcommand{\algorithmicrequire}{\textbf{输入:}}
	\renewcommand{\algorithmicensure}{\textbf{输出:}}
	\caption{深度优先遍历(dfs)}
	\label{alg_huanyuan2}
	\begin{algorithmic}[1]
		\REQUIRE 起始位置idx,归属簇标识mark,访问标识数组visited,元素数组plist,二维矩阵matrix,映射关系resmap
		\ENSURE 修改后的输入值
		\FOR {$ i=1 $ to $ n $}
		\IF {visited[i]为真}
		\STATE 该元素已经被划分好,不再继续
		\ENDIF
		\IF{matrix[idx][i] > 0}
		\STATE resmap[plist[i].cluId] = mark
		\STATE visited[i]=true
		\STATE dfs(i, mark, visited, plist,matrix, resmap)
		\ENDIF
		\ENDFOR
	\end{algorithmic}
\end{algorithm}

\begin{algorithm}[htbp]
	\renewcommand{\algorithmicrequire}{\textbf{输入:}}
	\renewcommand{\algorithmicensure}{\textbf{输出:}}
	\caption{还原结果}
	\label{alg_huanyuan1}
	\begin{algorithmic}[1]
		\REQUIRE 明文plist
		\ENSURE 聚类划分结果映射关系
		\STATE 构造二维数组matrix,大小为$ n\times n $存储相连关系
		\STATE 构建resmap存储初始簇id到结果簇id的映射
		\FOR {each p in plist}
		\STATE matrix[p.id1][p.id2] += p.isConn
		\ENDFOR
		\STATE 构造访问标识数组$ visited=[false]_n $
		\STATE mark = 1
		\FOR {$ i=1 $ to $ n $}
		\IF{visited[i]为真}
		\STATE 不再继续访问
		\ENDIF
		\STATE visited[i] = true
		\STATE mark += 1
		\STATE resmap[plist[i].cluId] = mark
		\STATE 执行dfs(i, mark, visited, plist,matrix, resmap)
		\ENDFOR
	\end{algorithmic}
\end{algorithm}
\section{改进的隐私保护dbscan}
\label{s4-t2}
本节我们将在上一节的基础上详细阐述一种改进的隐私保护DBSCAN方案,该方案能够在数据集被打乱的情况,使满足条件可以被划分到多个簇的数据点稳定分类到最近的簇中。

方案包含三个步骤,初始化、聚类以及还原结果。还原结果的过程与\ref{task1-huanyuan}一致,这里不多做解释。主要区别在于初始化和聚类过程,下面我们将着重介绍。

\subsection{初始化}
\begin{algorithm}[htbp]
	\renewcommand{\algorithmicrequire}{\textbf{输入:}}
	\renewcommand{\algorithmicensure}{\textbf{输出:}}
	\caption{初始化}
	\label{alg_t2_chushihua}
	\begin{algorithmic}[1]
		\REQUIRE 秘密共享的元素数组plist
		\ENSURE 距离关系$ u $
		\STATE 
	\end{algorithmic}
\end{algorithm}
\section{基于dbscan的隐私保护层次聚类}
\label{s4-t3}
\section{理论分析}
\label{s4-lilun}
\section{实验评估}
\label{s4-shiyan}
\section{本章小结}
\label{s4-xiaojie}